\documentclass[a4]{article}

\usepackage{fullpage}
\usepackage{graphicx}
\usepackage{tikz}
\usepackage{listings}
\usepackage{theorem}
\usepackage{url}
\usepackage{xspace}
\usepackage{amssymb}\usepackage{xtab}

\newcommand{\todo}[1]{{\bf [TODO: }\textit{#1}{\bf ]}}
\newcommand{\remark}[1]{{\bf [RemarkKVH: }\textit{#1}{\bf ]}}
\newcommand{\remarktmb}[1]{{\bf [RemarkTMB: }\textit{#1}{\bf ]}}

\newcommand{\EIS}{\textsf{EIS}\xspace}

\newcommand{\Goal}{\textsc{GOAL}\xspace}
\newcommand{\Jason}{\textsc{Jason}\xspace}
\newcommand{\twoAPL}{\textsc{2APL}\xspace}
\newcommand{\Jadex}{\textsc{Jadex}\xspace}

\newtheorem{example}{Example}





\begin{document}

\title{Environment Interface Standard for Agent-Oriented Programming \\
\textit{Multi-Agent Programming Competition 2007\\ Interface Description}
}

\author{Tristan Behrens}

\maketitle

In order to run the contest environment you have to download the package including the MASSim-server from
the Multi-Agent Contest homepage\footnote{\url{http://www.multiagentcontest.org}}. 
First start the server and then connect your agents.

\medskip\noindent{\textbf{Environment description:}} the environment is a grid-like, partially-accessible world. Miners are steered
by agents. The goal is to collect pieces of gold. More information is available at the 
contest homepage.

\medskip\noindent{\textbf{Jar-file:}} \texttt{eis-acconnector2007-0.2.jar} (included in the \EIS-package).

\medskip\noindent{\textbf{Entities:}}
\begin{description}
\item[\texttt{connector1},$\ldots$,\texttt{connector6}] each one is a connector to a single miner in the environment.
\end{description}

\noindent{\textbf{Types of entities:}} this interface does not take into account different types of entities.

\medskip\noindent{\textbf{Actions:}}
\begin{description}
\item[\texttt{connect(Identifier, Numeral , Identifier, Identifier)}] connects to the MASSim-server.
The first identifier is the hostname of the server. The numeral is its port. The second identifier denotes the user-name,
the final one denotes the password. This action has to be performed successfully in order to allow for other actions.
Example: \texttt{connect("139.174.100.201",12300,"agentred1","dfkj39")}.
\item[\texttt{right}, \texttt{up}, \texttt{left}, \texttt{down}] move the miner. 
\item[\texttt{pick} or \texttt{drop}] picks gold up or drops it.
\item[\texttt{mark(Identifier)} or \texttt{unmark}] mark a cell with an identifier or
remove a mark from a cell.
\item[\texttt{skip}] has no effect.
\end{description}

\noindent{\textbf{Percepts:}} all those percepts are both propagated as notifications and returned by the
\texttt{getAllPercepts}-method. Note that the interface implements a queue of percepts that is filled every time
a message from the MASSim-server is received and whose first entry is retrieved every time the \texttt{getAllPercepts}-method is called. The interface does not hold a world-model.
\begin{description}
\item[\texttt{connectionLost}] indicates that the connection to the server has been lost. 
\item[\texttt{simStart}] indicates that the simulation has begun.
\item[\texttt{depotPos(Numeral, Numeral)}] is the position of the depot. Example: \texttt{corralPos(10,10)}.
\item[\texttt{gridSize(Numeral, Numeral)}] represents the size of the grid. The first value is the width, the second one is 
the height. Example:\texttt{gridSize(100,100)}.
\item[\texttt{simId(Identifier)}] denotes the id of the simulation. Example: \texttt{simId("eldorado")}
\item[\texttt{opponent(Identifier)}] denotes the opponent in the current match. 
Example: \texttt{opponent("hillbillies")}
\item[\texttt{steps(Numeral)}] indicates how many steps the simulation lasts. Example: \texttt{steps(1000)}
\item[\texttt{simEnd}] indicates that the current simulation is over.
\item[\texttt{result(Identifier)}] represents the result of the simulation. Values could be either \texttt{win},  
\texttt{lose}, or  \texttt{draw}. Example: \texttt{result(win)}
%\item[\texttt{finalScore(Numeral)}] represents the final-score of the simulation. Example: \texttt{finalScore(42)}
\item[\texttt{bye}] indicates that the overall tournament is over.
\item[\texttt{cell(Identifier,Numeral,Numeral,Identifier,Numeral)}] denotes the content of a cell. 
The first identifier denotes the relative position (\texttt{e},\texttt{ne},\texttt{n},\texttt{nw},\texttt{w},\texttt{sw},\texttt{s},\texttt{se}). 
The first two numerals represent the absolute position of the cell. 
The second identifier represents the object. The final numeral represents the step of the simulation.
Possible values for the object-identifier are \texttt{agent}, \texttt{obstacle}, \texttt{gold}, \texttt{mark}, 
\texttt{depot}, \texttt{unknown}, and \texttt{empty}.
Example: \texttt{cell(nw,10,10,depot,123)}
%\item[\texttt{id(Numeral num)}] indicates the identifier of the \texttt{id(70)}
\item[\texttt{pos(Numeral,Numeral,Numeral)}] denotes the current position of the miner at a specific step. Example: \texttt{pos(10,15,123)}
\item[\texttt{currentItems(Numeral,Numeral)}] denotes the current number carried items at a specific step. Example: \texttt{currentItems(24,123)}
\item[\texttt{currentStep(Numeral num)}] indicates the current step of the simulation.
Example: \texttt{currentStep(123)}
\end{description}

\noindent{\textbf{Environment-management:}} not supported.

\end{document}